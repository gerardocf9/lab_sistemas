\section{Programa p8.asm}

Para esta practica se presentaron muchas dificultades, en su mayoria por las
diferencias entre sistemas operativos trabajados. Para salvaguardar estos
inconvenientes, no se pudo recurrir a una maquina virtual, dado que los equipos
con los que se contaba, portatiles, no tienen recursos suficientes. De igual
forma, los programas capaces de ejecutar codigo de windows en linux presentaron
muchos inconvenientes y, particularmente la depuracion, resultaban muy complejos.
Dado esto, se opto por hacer el programa funcional en linux y utilizar el
\comillas{calling convention de windows} para que sea funcional en este sistema
operativo; para poder realizar esto, se utilizo un codigo en c el cual hace un
llamamado a un procedimiento en assembler, este es la conversion del programa de
la practica 7, \Cite{pract7}, adaptado en forma de procedimiento y en lenguaje
ensamblador \comillas{NASM}, el cual corre nativo en linux, ya que ensambla en la
mayoria de plataformas, y por ende la
depuracion se hace con facilidad, usando \comillas{GDB} por ejemplo, se opto por
esta forma, ademas de por las limitaciones anteriormente nombradas, por los
conocimientos previos de este \comillas{lenguaje}.

El programa se desgloza en dos codigos, \comillas{\textbf{main.c}} y
\comillas{\textbf{makeNewDataBlock.asm}}.

\subsection*{main.c}
Es el programa principal, se encarga de hacer el llamado a \ref{asm}. Esta
diseñado de forma modular, posee 6 funciones ademas del main:

\begin{itemize}
    \item \textbf{\comillas{callMe}} la cual hace la llamada al codigo en assembler.
    \item  \textbf{\comillas{getInt}} funcion para la lectura de un entero.
    \item \textbf{\comillas{writeLittleEndian}} la cual se encarga de la escritura
        en little endian.
    \item \textbf{\comillas{readLittleEndian}}  que se encarga de leer en little
        endian.
    \item \textbf{\comillas{strfind}} la cual busca y devuelve una cadena de
        caracteres.
    \item \textbf{\comillas{readFile}} que lee un archivo.
    \item \textbf{\comillas{main}} es el main del programa y contiene la logica
        del mismo.

        \begin{itemize}
            \item Lee el nombre de un archivo.
            \item Abre el archivo ingresado, comprueba que cumpla todos los
                parametros, lee el tamaño, chunksize y se separan los datos de
                interes pertinentes al formato que se encuentran en el archivo,
                de la misma forma que en la practica anterior.
            \item Pide el valor de \comillas{m}.
            \item Llama a la funcion en assembler,
            \item Escribe la informacion en el nuevo archivo.
        \end{itemize}
\end{itemize}

\subsection*{newBlock.asm}\label{asm}

Se encarga de modificar un buffer de entrada con respecto a los requerimientos
dados, es decir el factor \comillas{m} de compresión que se requiera de forma
similar a los hecho en la practica 7.


\subsection*{Llamado de funciones de assembler desde C}

Los lenguajes compilados depende de un \textbf{ABI} el cual especifica
como serán llamadas las funciones en lenguaje maquina y usualmente
este depende en mayor medida del sistema operativo, aunque tambien puede
depender del compilador usado. Para los sistemas basados en x86 de 32 bits
lo usual era pasar los parametros de las funciones por el stack ya que
los registros en x86 eran bastante escasos pero en 64 bits tanto en linux,
como en windows se utilizan algunos registros para pasar parametros a las
funciones. El pasar valores por registro minimiza el uso de ram o cache, lo
cual mejora el performance de los programas en las plataformas de 64 bits.

Se utilizo para esta actividad el \textbf{ABI} de 64 bits tanto de linux
como de windows, sin embargo existen problemas con la version de ensamblador
para windows con el 5to parametro de la función por alguna razón no esta
pasando el parametro que deberia pasarse por el stack, si hace pop el programa
tira un \textbf{segfault}. Se intento depurar el programa de diferentes
formas pero ninguna fue fructifera, quizas es el compilador o algun problema
aun desconocido. En el caso de aplicaciones de 32 bits de windows se complica un
poco el llamado de funciones porque no existe un solo estandar sino que existen
3, por se decidio no utilizar.
