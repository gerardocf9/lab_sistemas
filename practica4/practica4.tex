%\documentclass[10pt]{article}

\section{Programa DOS}

El código comentado se adjunta en el archivo (dos/lectura.asm)

Las llamadas a sistema en \textbf{MS-DOS} funcionan mediante la interrupción
\verb|INT 21h| la cual recibe como parámetro el código de la función en el
registro \verb|AX|. Para la elaboración del programa se utilizaron 4 llamadas a
sistema, una para abrir el archivo, otra para leer, una para cerrar y
finalmente una para imprimir en pantalla.

La primera función fue \verb|716Ch| la cual sirve para crear o abrir un
archivo, toma 4 parámetros adicionales al código de función. En \verb|BX|
se pasa el tipo de acceso, en este caso se tomo 0 porque se va leer el archivo.
En \verb|CX| se pasan atributos como no se hará nada especial se paso 0 que significa
normal.  En \verb|DX| se pasa la acción a tomar, se pasa 1 porque se desea abrir
el archivo. Finalmente en \verb|DS:SI| se pasa un puntero al nombre del
archivo. La función retorna en AX el handle del archivo.

La segunda función que se utiliza es \verb|3Fh| la cual lee un arreglo de
bytes de un archivo o dispositivo. Toma tres parámetros, en \verb|BX|
se pasa el handle del archivo, en \verb|CX| el numero máximo de bytes a leer
y finalmente en \verb|DS:DX| la dirección del buffer a escribir. En este
programa se toman 4kb de salida, mas de ahí el programa truncara, pero
es mas que suficiente para un programa que muestre en pantalla un archivo.
El programa retorna la cantidad de bytes leídos del archivo y se guarda en una
variable.

La tercera función que se utiliza es \verb|3Eh| la cual cierra un archivo.
Solo toma un parámetro en \verb|BX| el handle del archivo a cerrar.

Finalmente la ultima función utilizada fue \verb|40h| la cual permite
escribir un arreglo de bytes en un archivo o dispositivo. Toma tres parámetros
en \verb|BX| el handle del archivo o dispositivo a escribir, en este
caso pasamos 1 porque es el handle de la salida estándar. En \verb|CX|
se pasa la cantidad de bytes a escribir que es igual a la cantidad de
bytes leídos del archivo. Finalmente pasamos en \verb|DS:DX| el buffer
que contiene la información leída con anterioridad.


\lstinputlisting[language={[x86masm]Assembler}]{practica4/dos/lectura.asm}

\section{Programa consola}

El código comentado se adjunta en el archivo (consola/lectura.asm)

\lstinputlisting[language={[x86masm]Assembler}]{practica4/consola/lectura.asm}

Inicialmente es directivas e inclusión de librerías, luego viene el prototipo
de funciones y el macro que permiten el controlar la entrada y salida de datos.
Cabe resaltar que se utilizo como modelo la practica 2, posteriormente,
vienen los segmentos de data y código:

En el segmento de data inicializada \comillas{\textbf{.DATA}} se encuentran:

\begin{itemize}
    \item \comillas{Msg1} el cual es un mensaje que indica al usuario cuando
        debe ingresar un path o el nombre del archivo.

    \item \comillas{Msg4} mensaje que indica la salida del programa.

    \item \comillas{CRLF} string dedicada a dejar una linea en blanco.
\end{itemize}

\vspace{0.5cm}

En el segmento de data sin inicializar \comillas{\textbf{.DATA?}} se encuentran:

\begin{itemize}
    \item \comillas{inbuf} buffer para la lectura de datos.

    \item \comillas{hfile} \comillas{puntero} a la dirección de la fila que se
        leerá.

    \item \comillas{FileSize} espacio para el tamaño, en bytes, del fichero.

    \item \comillas{hMem} \comillas{puntero} a la memoria creada con el contenido
        del fichero.

    \item \comillas{BytesRead} cantidad de bytes que se leyeron.

\end{itemize}

\vspace{0.5cm}

Posteriormente, comienza el código que básicamente es ejecutar \comillas{syscall}
a través de la directiva \comillas{INVOKE}.


Primero se pide por consola y se lee por la misma el nombre o path de la fila
que se abrirá. Una vez recibido se usa la api de windows con \comillas{CreateFile} para
abrir la fila \Cite{MicrosoftCreateFileA}, este devuelve la dirección en \comillas{eax}
por lo tanto se mueve a hfile. Luego, se utiliza la api nuevamente con \comillas{GetFileSize}
\Cite{Windowsgetsize} y se mueve el valor devuelto de \comillas{eax} a
\comillas{FileSize}, se incrementa el valor de \comillas{eax} para al pedir
memoria dinámicamente, necesario si la fila es muy grande, poder agregar el
carácter de terminación y usar de forma segura la función de escritura.


Se pide la memoria dinámica con la api \comillas{GlobalAlloc} \Cite{GlobalAlloc},
se guarda en \comillas{hMem}, se le da el valor del tamaño de la fila de nuevo a
\comillas{eax} para poder usarlo como puntero y darle el valor \comillas{0}
(carácter de terminación) a \comillas{hMem} (espacio para el contenido de la
fila + carácter terminación).


Finalmente se lee el contenido con la función de la api ReadFile
\Cite{MicrosoftReadFile} se cierra la fila con \comillas{CloseHandle}
\Cite{CloseHandle}
luego se imprime por pantalla el contenido de la fila. Se libera la memoria con
\comillas{GlobalFree} y para que se visualice correctamente el resultado, se
imprime un salto de linea y se manda a leer esperando por un enter para finalizar.


El programa termina con \comillas{Invoke ExitProcess} para terminar la ejecución
correctamente.

\section{Programa ventana}

El código comentado se adjunta en el archivo (ventana/DIALOG.asm y ventana/recursos.rc)

\lstinputlisting[language={[x86masm]Assembler}]{practica4/ventana/DIALOG.asm}


Para este programa se utilizo como base los tutoriales de \comillas{Iczelion}
\Cite{Iczelion} mas específicamente el 11-1, ya que en la carpeta 11-3 estaba
prácticamente el código hecho y se quería intentar hacer a cuenta propia.


Esta estructura da acceso a una ventana principal con un submenu superior, desde
el cual se puede cerrar el programa y se puede abrir una ventana emergente con un
cuadro de texto y 2 botones. Aquí es en donde se basaron todas las modificaciones.


Para el funcionamiento del código, se incluyo la librería \comillas{masm32} y se
declararon variables adicionales en la sección \comillas{.data?}


\begin{itemize}
    \item \comillas{inbuf} buffer para la lectura de datos.

    \item \comillas{hfile} \comillas{puntero} a la dirección de la fila que se
        leerá.

    \item \comillas{FileSize} espacio para el tamaño, en bytes, del fichero.

    \item \comillas{hMem} \comillas{puntero} a la memoria creada con el contenido
        del fichero.

    \item \comillas{BytesRead} cantidad de bytes que se leyeron.

\end{itemize}



Cabe resaltar que el programa cuenta de estructuras que no se modificaron ya que
parecían apropiadas, estas son constantes y el nombre de las ventanas. El segmento
 de código hace uso de 4 invoke, el primero es para poder manejar los módulos
 \comillas{GetModuleHandle}, el segundo \comillas{GetCommandLine} permite hacer
 uso de una \comillas{CMD}, el tercero, \comillas{WinMain} le da el control del
 programa a un subproceso descrito posteriormente y el final para cerrar los procesos.



\comillas{Invoke WinMain} le da el control del programa a un subproceso y le pasa
como parámetro las variables hInstance, NULL, CommandLine, SW\_SHOWDEFAULT, con
las cuales se creara la ventana main, estas se inicializan y posteriormente se
crea un objeto, struct, y a el se le asignan los, valores, variables y procedimientos
requeridos, el tamaño de la ventana, el color, los nombres de los objetos gráficos
estilo de cursor entre otros, el mas importante es
\comillas{mov   wc.lpfnWndProc, OFFSET WndProc}, aquí se le da a la clase la
dirección de otro subproceso definido y explicado posteriormente, en el cual se
hará todo el proceso de búsqueda, lectura y escritura del fichero. \\
Después de estas definiciones se usan las directivas \comillas{INVOKE ShowWindow}
y \comillas{INVOKE UpdateWindow} que crean una ventana y la actualizan con toda
la información proveída desde la clase.\\
Posteriormente, se entra en un ciclo while true el cual hace la función de
una especie de \comillas{framework} ya que mantiene el programa en ejecución a la
espera de que un evento suceda y sea necesario realizar alguna acción, en este caso
\comillas{GetMessage} se encarga de verificar si ocurre algún evento,
\comillas{eax} guarda cualquier posible cambio y si es 0 indica la finalización
del programa.


\comillas{WindProc} es otro proceso construido y es llamado automáticamente por la
clase (ya que se le asigno su dirección, y es llamado cuando es requerido) y, básicamente,
se encarga de revisar si algún evento sucedió, y si sucedió enviar los mensajes con
\comillas{PostQuitMessage} si se destruyo la ventana, para finalizar el proceso
mencionado anteriormente ( asigna el valor 0 a \comillas{eax}), cuando este no es el
caso, se trata de una serie de condicionales anidados para ir revisando todos los
posibles eventos y tomar las medidas necesarias en cada caso. Estos son:

\begin{itemize}
    \item  Si \comillas{uMsg$==$WM\_DESTROY}, condición de terminación, se envía el
        mensaje 0 (null) en \comillas{eax} al \comillas{framework} (\comillas{.while true})
        para terminar su ejecución.

    \item Si no y \comillas{uMsg$==$WM\_COMMAND}, hace la comparación de la entrada
        al submenu.

        \begin{itemize}
            \item Si \comillas{ax$==$IDM\_ABOUT} crea la nueva ventana y le da la
                 dirección a otro proceso definido posteriormente donde se realizan
                todo lo relacionado a el
                el \comillas{InputBox} y los botones que permiten ingresar el
                nombre del fichero. Cabe resaltar que \comillas{IDM\_ABOUT} no
                es mas que el identificador de la ventana emergente creada que
                contendrá todos los elementos anteriormente nombrados, este
                nombre puede ser cambiado, requeriría modificar aquí y en
                el archivo \comillas{recursos.rc} ya que aquí se define el
                \comillas{InputBox}.

            \item Si no, la única otra opción disponible implica cerrar, así que
                llama al proceso \comillas{DestroyWindow}

        \end{itemize}

    \item En el caso que no se haya cumplido nada de esto, llama a
        \comillas{DefWindowProc} para definir la ventana y regresa el control.
\end{itemize}



El proceso \comillas{DlgProc} es igualmente una secuencia de condicionales anidados:

\begin{itemize}
    \item \comillas{.if iMsg$==$WM\_INITDIALOG} se inicializa la ventana de diálogos
        y se centra la atención en \comillas{eax}.
        \comillas{inputbox}

    \item \comillas{.elseif iMsg$==$WM\_CLOSE} el mensaje es para cerrar la pestaña.
        se clickeo el salir de la ventana.

    \item \comillas{.elseif iMsg$==$WM\_COMMAND} Ha ocurrido algún evento dentro de
    la ventana, particularmente se trabajan los botones:
        \begin{itemize}
            \item \comillas{.if eax$==$IDC\_EXIT} se presiono el botón \comillas{salir}
                 y se cierra la ventana, mandando un mensaje de terminación con
                 \comillas{SendMessage}.

             \item \comillas{.elseif eax$==$IDC\_BUTTON} se presiono el botón para
                 la lectura del texto. entonces se hace:
                 \begin{itemize}
                    \item \comillas{GetDlgItemText} para obtener el texto en el
                         \comillas{InputBox} y guardarlo en \comillas{inbuf}.

                    \item \comillas{CreateFile} para abrir la fila especificada
                         en \comillas{inbuf}.

                    \item \comillas{GetFileSize} tomar el tamaño de la fila.

                    \item \comillas{GlobalAlloc} reservar memoria dinámica. Se
                        hacen las mismas consideraciones que en el programa consola.

                    \item \comillas{ReadFile} lee la fila.

                    \item \comillas{CloseHandle} cierra la fila.

                    \item \comillas{StdOut} escribe el contenido en \comillas{hMem}.

                    \item \comillas{MessageBox} Manda una ventana emergente con
                        el contenido del fichero.

                    \item \comillas{GlobalFree} libera la memoria pedida.

                 \end{itemize}

        \end{itemize}

    \item Si no se ejecuta ninguna de estas opciones, se devuelve \comillas{False}
        en \comillas{eax}.

\end{itemize}

Si algo de eso ocurrió, devuelve \comillas{True} en \comillas{eax}.

Con ese ultimo procedimiento finaliza todo el código.
