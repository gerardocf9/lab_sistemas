%\documentclass[10pt]{article}

\section{Programa lectura.asm}

El código del programa se encuentra en un único archivo llamado \textbf{lectura.asm}.
Este programa funciona mediante una aplicación de consola usando la api de 32
bits de Windows. Para la lectura del archivo se usan las funciones
CreateFile en modo lectura, GetFileSize y ReadFile el tamaño.
Se utiliza una estructura de decisión conformada por \comillas{cmp} y \comillas{je}
y se divide el programa en 2 grandes partes.GetFileSize
se utiliza para crear un bloque de memoria lo suficientemente grande
para almacenar el archivo en memoria, mas el nuevo registro. Para escribir
el archivo una vez modificado se utilizan lstrlen para poder saber la cantidad
de bytes a escribir y finalmente se usa WriteFile. Además se crearon
procedimientos para facilitar la legibilidad y escritura, se crean strings,
guardan y escriben de esta forma.


El programa se puede dividir en dos grandes subprogramas y un previo común:

Cabe resaltar que se utilizo la misma estructura que en \cite{pract5} con algunas
modificaciones, creación de procedimientos con código reutilizable, y se le
agrego:


\begin{itemize}
    \item Estructura de información y decisión para indicar cual subprograma hacer.

    \item Encapsulación de la practica anteriormente nombrada para que su
        ejecución sea controlada por la entrada del usuario.

    \item Sub programa que se encarga de mostrar el archivo seleccionado indicando
        el numero de linea correspondiente al registro a imprimir y la opción de
        seleccionar y eliminar uno de estos.
\end{itemize}


\subsection*{Selección del programa a ejecutar}
Para la elección de que programa se ejecutara, se muestra por pantalla la
información: \comillas{Please type 1: to insert a new camp 2: to delete a existing line}
y se utiliza el código de la practica 5 en forma de procedimiento para convertir
la entrada en un entero, este numero se compara y se presentan 3 posibilidades:

\begin{itemize}
        \item Se ingreso 1. Se ejecuta la sección del código asociada, ya creado
            en \cite{pract5} sin ninguna modificación.

        \item Se ingreso 2. Se ejecuta la sección de código propia de esta
            funcionalidad, se explicara posteriormente

        \item se ingreso otro valor. Se ejecuta una sección de código adicional,
            esta simplemente indica que el numero es una opción invalida y
            termina la ejecución
\end{itemize}

\subsection*{Eliminación de una linea}
Esta funcionalidad es la implementación solicitada, para esto se creo una sección
de código especifica. Lo primero que se realiza es mostrar por pantalla el numero
de linea-registro en el que se encuentre de la forma \comillas{(n)}, con \comillas{n}
siendo la fila correspondiente del registro en el archivo.
Esto se logra al separa el archivo en cada fin de linea, como se hizo anteriormente
para contar la cantidad de lineas. se aprovecha este proceso de conteo para indicar
en que linea se encuentra. para esto se utiliza un procedimiento  y se convierte
un numero entero, el contador de lineas, en un string, se guarda en un buffer
auxiliar y se imprime, entre la impresión de los símbolos \comillas{(} y
\comillas{)} y luego se imprime la linea correspondiente, para esto se aprovecha
que:
$$ N_{bytes} = DIR_{fin} - DIR_{inicio}$$
por lo tanto, solo hace falta el guardar la dirección inicial constantemente,
primero es \comillas{hmem} cuando se consigue la siguiente linea, esta seria la
direccion final de esa linea y la nueva direccion inicial.

Usando esto se construyo
un procedimiento el cual copia \comillas{n} char de un string de entrada a un buffer
de salida; este es llamado dentro de un procedimiento de impresión al cual se le
pasan las direcciones antes mencionadas, hace el calculo de la cantidad de caracteres
crea la string y la imprime.


Una vez terminada la impresión, se pide ingresar la linea correspondiente al
registro a eliminar y se procede a esto. Este segmento de código es igual al hecho
en la practica 5.

Para \comillas{eliminar} la linea simplemente se deja de escribir en el archivo,
para esto se toma ventaja de la impresión selectiva que se puede hacer con la
función de la api de Windows \comillas{WriteFile}, de esta forma solo se elige
que escribir enviando las direcciones de inicio de escritura y la cantidad de
bytes a escribir adecuados. El procedimiento para esto es similar al explicado
anteriormente solo que se envían a escribir en el archivo seleccionado  en vez
de a mostrar por pantalla.















